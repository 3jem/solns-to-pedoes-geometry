% Created 2019-06-19 Wed 12:07
% Intended LaTeX compiler: pdflatex
\documentclass[11pt]{article}
\usepackage[utf8]{inputenc}
\usepackage[T1]{fontenc}
\usepackage{graphicx}
\usepackage{grffile}
\usepackage{longtable}
\usepackage{wrapfig}
\usepackage{rotating}
\usepackage[normalem]{ulem}
\usepackage{amsmath}
\usepackage{textcomp}
\usepackage{amssymb}
\usepackage{capt-of}
\usepackage{hyperref}
\usepackage{framed}
\usepackage{parskip}
\usepackage{subfig}
\usepackage{hyperref}
\usepackage[utf8]{inputenc}
\author{Joel}
\date{\today}
\title{Solutions to Exercises 0.1 to 0.6 of Pedoe's "Geometry: A Comprehensive Course"}
\hypersetup{
 pdfauthor={Joel},
 pdftitle={Solutions to Exercises 0.1 to 0.6 of Pedoe's "Geometry: A Comprehensive Course"},
 pdfkeywords={},
 pdfsubject={},
 pdfcreator={Emacs 26.2 (Org mode 9.1.9)}, 
 pdflang={English}}
\begin{document}

\maketitle

\section*{0.1}
\label{sec:orge4f8cf0}
\subsection*{Prove \(\overline{z_1 + z_2 + ... + z_n} = \overline{z_1} + \overline{z_2} + ... + \overline{z_n}\)}
\label{sec:orgcb15ff6}
Base Case:

Let \(z_1 = (a_1, b_1)\), \(z_2 = (a_2, b_2)\)

Expand \(\overline{z_1 + z_2}\)

\begin{align*} 
\overline{z_1 + z_2} &= \overline{(a_1, b_1) + (a_2, b_2)} \\
                     &= \overline{(a_1 + a_2, b_1 + b_2)} \\
                     &= (a_1 + a_2, -b_1 - b_2) \\
		     &= (a_1, -b_1) + (a_2, -b_2) \\
		     &= \overline{(a_1, b_1)} + \overline{(a_2, b_2)} \\
		     &= \overline{z_1} + \overline{z_2}
\end{align*}

Inductive Case:

Assuming that for any k complex numbers, \(\overline{z_1 + z_2 + ... + z_k} = \overline{z_1} + \overline{z_2} + ... + \overline{z_k}\) holds, we will show that \(\overline{z_1 + z_2 + ... + z_k + z_{k+1}} = \overline{z_1} + \overline{z_2} + ... + \overline{z_k} + \overline{z_{k+1}}\) also holds. 

Let \(z'_k = z_k + z_{k+1}\)

By the inductive hypothesis,

\(\overline{z_1 + z_2 + ... + z'_k} = \overline{z_1} + \overline{z_2} + ... + \overline{z'_k}\)

By the base case, \(\overline{z'_k} = \overline{z_k + z_{k+1}} = \overline{z_k} + \overline{z_{k+1}}\)

Therefore,

\(\overline{z_1 + z_2 + ... + z_k + z_{k+1}} = \overline{z_1} + \overline{z_2} + ... + \overline{z_k} + \overline{z_{k+1}}\)

QED

\subsection*{Prove \(\overline{z_1 \cdot z_2 \cdot ... \cdot z_n} = \overline{z_1} \cdot \overline{z_2} \cdot ... \cdot \overline{z_n}\)}
\label{sec:org8630e47}
Base Case:

Let \(z_1 = (a_1, b_1)\), \(z_2 = (a_2, b_2)\)

Expand \(\overline{z_1 \cdot z_2}\)

\begin{align*} 
\overline{z_1 \cdot z_2} &= \overline{(a_1, b_1) \cdot (a_2, b_2)} \\
                         &= \overline{(a_1a_2 - b_1b_2, a_1b_2 + b_1a_2)} \\
                         &= (a_1a_2 - b_1b_2, -a_1b_2 - b_1a_2)
\end{align*}

Expand \(\overline{z_1} \cdot \overline{z_2}\)

\begin{align*}
\overline{z_1} \cdot \overline{z_2} &= (a_1, -b_1) \cdot (a_2, -b_2) \\
                                    &= (a_1a_2 - b_1b_2, -a_1b_2 - b_1a_2)
\end{align*}

so \[ \overline{z_1 \cdot z_2} = \overline{z_1} \cdot \overline{z_2} \]

Inductive Case:

Assuming that for any k complex numbers, \(\overline{z_1 \cdot z_2 \cdot ... \cdot z_k} = \overline{z_1} \cdot \overline{z_2} \cdot ... \cdot \overline{z_k}\) holds, we will show that \(\overline{z_1 \cdot z_2 \cdot ... \cdot z_k \cdot z_{k+1}} = \overline{z_1} \cdot \overline{z_2} \cdot ... \cdot \overline{z_k} \cdot \overline{z_{k+1}}\) also holds.

Let \(z'_k = z_k \cdot z_{k+1}\)

By the inductive hypothesis,

\(\overline{z_1 \cdot z_2 \cdot ... \cdot z'_k} = \overline{z_1} \cdot \overline{z_2} \cdot ... \cdot \overline{z'_k}\)

From the base case, \(\overline{z'_k} = \overline{z_k \cdot z_{k+1}} = \overline{z_k} \cdot \overline{z_{k+1}}\)

Therefore,

\(\overline{z_1 \cdot z_2 \cdot ... \cdot z_k \cdot z_{k+1}} = \overline{z_1} \cdot \overline{z_2} \cdot ... \cdot \overline{z_k} \cdot \overline{z_{k+1}}\)

QED

\section*{0.2 Show \(|z_1 + z_2 + ... + z_n| \leq |z_1| + |z_2| + ... + |z_n|\)}
\label{sec:org9d0bb7a}
We will show this by induction on the number of complex numbers involved.

Base Case:

\(|z_1 + z_2| \leq |z_1| + |z_2|\) is true by the triangle inequality.

Inductive Case:

Assuming \(|z_1 + z_2 + ... + z_k| \leq |z_1| + |z_2| + ... + |z_k|\) holds, we will show that \(|z_1 + z_2 + ... + z_k + z_{k+1}| \leq |z_1| + |z_2| + ... + |z_k| + |z_{k+1}|\) also holds.

Let \(|z'_k| = |z_k| + |z_{k+1}|\) then by the inductive hypothesis \(|z_1 + z_2 + ... + z'_k| \leq |z_1| + |z_2| + ... + |z'_k|\). By the base case, \(|z'_k| = |z_k + z_{k+1}| \leq |z_k| + |z_{k+1}|\) therefore \(|z_1 + z_2 + ... + z_k + z_{k+1}| \leq |z_1| + |z_2| + ... + |z_k| + |z_{k+1}|\).

\section*{0.3 Prove |z\(_{\text{1}}\) + z\(_{\text{2}}\)|\(^{\text{2}}\) + |z\(_{\text{1}}\) - z\(_{\text{2}}\)|\(^{\text{2}}\) = 2(|z\(_{\text{1}}\)|\(^{\text{2}}\) + |z\(_{\text{2}}\)|\(^{\text{2}}\))}
\label{sec:org9c7dbae}
LHS
\begin{align*}
& |z_1 + z_2|^2 + |z_1 - z_2|^2 \\
& (z_1 + z_2)(\overline{z_1} + \overline{z_2}) + (z_1 - z_2)(\overline{z_1} - \overline{z_2}) \\
& 2(|z_1|^2 + |z_2|^2)
\end{align*}

which equals the RHS.

QED

The formula can be interpreted as two right angle triangles with the same hypotenuse. The RHS describes a right angle triangle that is always an isosceles triangle with sides \(\sqrt{|z_1|^2 + |z_2|^2}\), \(\sqrt{|z_1|^2 + |z_2|^2}\), \(\sqrt{2(|z_1|^2 + |z_2|^2)}\) and the LHS describes a triangle with sides \(|z_1 + z_2|\), \(|z_1 - z_2|\), \(\sqrt{|z_1 + z_2|^2 + |z_1 - z_2|^2}\).

\section*{0.4}
\label{sec:org6468853}
\subsection*{Show \((z_1 - z_4) \cdot (z_2 - z_3) + (z_2 - z_4) \cdot (z_3 - z_1) + (z_3 - z_4) \cdot (z_1 - z_2) = 0\)}
\label{sec:org7340ffc}
Expanding this out and evaluating results in 0.

\subsection*{Show \(|AD||BC| - |BD||CA| - |CD||AB| \leq 0\).}
\label{sec:orgf2e04e6}
\(|AD||BC| + |BD||CA| + |CD||AB|\) can be represented by \((z_1 - z_4) \cdot (z_2 - z_3) + (z_2 - z_4) \cdot (z_3 - z_1) + (z_3 - z_4) \cdot (z_1 - z_2)\) which we showed equals 0.

Since \(|AD||BC| - |BD||CA| - |CD||AB| \leq |AD||BC| + |BD||CA| + |CD||AB| = 0\) and magnitudes are always non-negative, \(|AD||BC| - |BD||CA| - |CD||AB| \leq 0\).

\section*{0.5 Express \((a_1^2 + b_1^2)(a_2^2 + b_2^2)\) as \(a^2 + b^2\) using \(|z_1 \cdot z_2| = |z_1| \cdot |z_2|\)}
\label{sec:orga477e08}
Let \(z_1 = a_1 + b_1i\) and \(z_2 = a_2 + b_2i\)

\begin{align*}
(a_1^2 + b_1^2)(a_2^2 + b_2^2) &= |z_1|^2 |z_2|^2 \\
       	 	      	       &= |z_1||z_1||z_2||z_2| \\
			       &= |z_1 \cdot z_1 \cdot z_2 \cdot z_2| \\
			       &= |z_1 \cdot z_2|^2 \\
			       &= |a_1a_2 + a_2b_1i + a_1b_2i - b_1b_2|^2 \\
			       &= |(a_1a_2 - b_1b_2) + (a_1b_2 + a_2b_1)i|^2 \\
			       &= (a_1a_2 - b_1b_2)^2 + (a_1b_2 + a_2b_1)^2
\end{align*}

\section*{0.6}
\label{sec:orge9b2f4c}
\subsection*{Show that the equation of a circle in the plane can be described by the equation \(|z - z_0| &= r^2\).}
\label{sec:org71c7a7e}
First, we will represent the x and y position of points on a circle by the real and imaginary parts of a complex number.

\[ (x - x_0) + (y - y_0)i \]

Then we will constrain the equation so that the magnitude squared of the complex number equals r\(^{\text{2}}\) the radius of the circle squared.

\[ |(x - x_0) + (y - y_0)i|^2 = r^2 \]

Let \(z = x+yi\) and \(z_0 = x_0 + y_0i\)

\begin{align*}
  |z - z_0| &= r^2 \\
  ((x-x_0) + (y - y_0)i)((x-x_0) - (y-y_0)i) &= r^2 \\
  (x-x_0)^2 + (y-y_0)^2 &= r^2
\end{align*}

which is the traditional equation of a circle in a plane.

\subsection*{Show that \(|z|^2 - z\overline{z_0} - z_0\overline{z} + z_0\overline{z_0} - r^2 = 0\) is another valid equation describing a circle in a plane.}
\label{sec:orgb9e9fc7}

\begin{align*}
  |z - z_0|^2 &= r^2 \\
  (z - z_0)(\overline{z} - \overline{z_0}) &= r^2 \\
  z \overline{z} - z\overline{z_0} - z_0\overline{z} + z_0\overline{z_0} &= r^2 \\
  |z|^2 - z\overline{z_0} - z_0\overline{z} + z_0\overline{z_0} - r^2 &= 0
\end{align*}
\end{document}